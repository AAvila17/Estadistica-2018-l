\documentclass[]{article}
\usepackage{lmodern}
\usepackage{amssymb,amsmath}
\usepackage{ifxetex,ifluatex}
\usepackage{fixltx2e} % provides \textsubscript
\ifnum 0\ifxetex 1\fi\ifluatex 1\fi=0 % if pdftex
  \usepackage[T1]{fontenc}
  \usepackage[utf8]{inputenc}
\else % if luatex or xelatex
  \ifxetex
    \usepackage{mathspec}
  \else
    \usepackage{fontspec}
  \fi
  \defaultfontfeatures{Ligatures=TeX,Scale=MatchLowercase}
\fi
% use upquote if available, for straight quotes in verbatim environments
\IfFileExists{upquote.sty}{\usepackage{upquote}}{}
% use microtype if available
\IfFileExists{microtype.sty}{%
\usepackage{microtype}
\UseMicrotypeSet[protrusion]{basicmath} % disable protrusion for tt fonts
}{}
\usepackage[margin=1in]{geometry}
\usepackage{hyperref}
\hypersetup{unicode=true,
            pdfborder={0 0 0},
            breaklinks=true}
\urlstyle{same}  % don't use monospace font for urls
\usepackage{color}
\usepackage{fancyvrb}
\newcommand{\VerbBar}{|}
\newcommand{\VERB}{\Verb[commandchars=\\\{\}]}
\DefineVerbatimEnvironment{Highlighting}{Verbatim}{commandchars=\\\{\}}
% Add ',fontsize=\small' for more characters per line
\usepackage{framed}
\definecolor{shadecolor}{RGB}{248,248,248}
\newenvironment{Shaded}{\begin{snugshade}}{\end{snugshade}}
\newcommand{\KeywordTok}[1]{\textcolor[rgb]{0.13,0.29,0.53}{\textbf{#1}}}
\newcommand{\DataTypeTok}[1]{\textcolor[rgb]{0.13,0.29,0.53}{#1}}
\newcommand{\DecValTok}[1]{\textcolor[rgb]{0.00,0.00,0.81}{#1}}
\newcommand{\BaseNTok}[1]{\textcolor[rgb]{0.00,0.00,0.81}{#1}}
\newcommand{\FloatTok}[1]{\textcolor[rgb]{0.00,0.00,0.81}{#1}}
\newcommand{\ConstantTok}[1]{\textcolor[rgb]{0.00,0.00,0.00}{#1}}
\newcommand{\CharTok}[1]{\textcolor[rgb]{0.31,0.60,0.02}{#1}}
\newcommand{\SpecialCharTok}[1]{\textcolor[rgb]{0.00,0.00,0.00}{#1}}
\newcommand{\StringTok}[1]{\textcolor[rgb]{0.31,0.60,0.02}{#1}}
\newcommand{\VerbatimStringTok}[1]{\textcolor[rgb]{0.31,0.60,0.02}{#1}}
\newcommand{\SpecialStringTok}[1]{\textcolor[rgb]{0.31,0.60,0.02}{#1}}
\newcommand{\ImportTok}[1]{#1}
\newcommand{\CommentTok}[1]{\textcolor[rgb]{0.56,0.35,0.01}{\textit{#1}}}
\newcommand{\DocumentationTok}[1]{\textcolor[rgb]{0.56,0.35,0.01}{\textbf{\textit{#1}}}}
\newcommand{\AnnotationTok}[1]{\textcolor[rgb]{0.56,0.35,0.01}{\textbf{\textit{#1}}}}
\newcommand{\CommentVarTok}[1]{\textcolor[rgb]{0.56,0.35,0.01}{\textbf{\textit{#1}}}}
\newcommand{\OtherTok}[1]{\textcolor[rgb]{0.56,0.35,0.01}{#1}}
\newcommand{\FunctionTok}[1]{\textcolor[rgb]{0.00,0.00,0.00}{#1}}
\newcommand{\VariableTok}[1]{\textcolor[rgb]{0.00,0.00,0.00}{#1}}
\newcommand{\ControlFlowTok}[1]{\textcolor[rgb]{0.13,0.29,0.53}{\textbf{#1}}}
\newcommand{\OperatorTok}[1]{\textcolor[rgb]{0.81,0.36,0.00}{\textbf{#1}}}
\newcommand{\BuiltInTok}[1]{#1}
\newcommand{\ExtensionTok}[1]{#1}
\newcommand{\PreprocessorTok}[1]{\textcolor[rgb]{0.56,0.35,0.01}{\textit{#1}}}
\newcommand{\AttributeTok}[1]{\textcolor[rgb]{0.77,0.63,0.00}{#1}}
\newcommand{\RegionMarkerTok}[1]{#1}
\newcommand{\InformationTok}[1]{\textcolor[rgb]{0.56,0.35,0.01}{\textbf{\textit{#1}}}}
\newcommand{\WarningTok}[1]{\textcolor[rgb]{0.56,0.35,0.01}{\textbf{\textit{#1}}}}
\newcommand{\AlertTok}[1]{\textcolor[rgb]{0.94,0.16,0.16}{#1}}
\newcommand{\ErrorTok}[1]{\textcolor[rgb]{0.64,0.00,0.00}{\textbf{#1}}}
\newcommand{\NormalTok}[1]{#1}
\usepackage{graphicx,grffile}
\makeatletter
\def\maxwidth{\ifdim\Gin@nat@width>\linewidth\linewidth\else\Gin@nat@width\fi}
\def\maxheight{\ifdim\Gin@nat@height>\textheight\textheight\else\Gin@nat@height\fi}
\makeatother
% Scale images if necessary, so that they will not overflow the page
% margins by default, and it is still possible to overwrite the defaults
% using explicit options in \includegraphics[width, height, ...]{}
\setkeys{Gin}{width=\maxwidth,height=\maxheight,keepaspectratio}
\IfFileExists{parskip.sty}{%
\usepackage{parskip}
}{% else
\setlength{\parindent}{0pt}
\setlength{\parskip}{6pt plus 2pt minus 1pt}
}
\setlength{\emergencystretch}{3em}  % prevent overfull lines
\providecommand{\tightlist}{%
  \setlength{\itemsep}{0pt}\setlength{\parskip}{0pt}}
\setcounter{secnumdepth}{0}
% Redefines (sub)paragraphs to behave more like sections
\ifx\paragraph\undefined\else
\let\oldparagraph\paragraph
\renewcommand{\paragraph}[1]{\oldparagraph{#1}\mbox{}}
\fi
\ifx\subparagraph\undefined\else
\let\oldsubparagraph\subparagraph
\renewcommand{\subparagraph}[1]{\oldsubparagraph{#1}\mbox{}}
\fi

%%% Use protect on footnotes to avoid problems with footnotes in titles
\let\rmarkdownfootnote\footnote%
\def\footnote{\protect\rmarkdownfootnote}

%%% Change title format to be more compact
\usepackage{titling}

% Create subtitle command for use in maketitle
\newcommand{\subtitle}[1]{
  \posttitle{
    \begin{center}\large#1\end{center}
    }
}

\setlength{\droptitle}{-2em}
  \title{}
  \pretitle{\vspace{\droptitle}}
  \posttitle{}
  \author{}
  \preauthor{}\postauthor{}
  \date{}
  \predate{}\postdate{}


\begin{document}

\subsection{Practica 1}\label{practica-1}

\subsubsection{Nombre : Alex Avila Santos
20160332F}\label{nombre-alex-avila-santos-20160332f}

\subsubsection{Pregunta1:}\label{pregunta1}

\paragraph{Item a) El código muestra una secuencia dee valores del 5 al
-11 que progresa en pasos tamaño
0.3.}\label{item-a-el-codigo-muestra-una-secuencia-dee-valores-del-5-al--11-que-progresa-en-pasos-tamano-0.3.}

\begin{Shaded}
\begin{Highlighting}[]
\NormalTok{v <-}\StringTok{ }\KeywordTok{seq}\NormalTok{(}\DataTypeTok{from=}\DecValTok{5}\NormalTok{,}\DataTypeTok{to=}\OperatorTok{-}\DecValTok{11}\NormalTok{,}\DataTypeTok{by=}\OperatorTok{-}\FloatTok{0.3}\NormalTok{)}
\NormalTok{v}
\end{Highlighting}
\end{Shaded}

\begin{verbatim}
##  [1]   5.0   4.7   4.4   4.1   3.8   3.5   3.2   2.9   2.6   2.3   2.0
## [12]   1.7   1.4   1.1   0.8   0.5   0.2  -0.1  -0.4  -0.7  -1.0  -1.3
## [23]  -1.6  -1.9  -2.2  -2.5  -2.8  -3.1  -3.4  -3.7  -4.0  -4.3  -4.6
## [34]  -4.9  -5.2  -5.5  -5.8  -6.1  -6.4  -6.7  -7.0  -7.3  -7.6  -7.9
## [45]  -8.2  -8.5  -8.8  -9.1  -9.4  -9.7 -10.0 -10.3 -10.6 -10.9
\end{verbatim}

\paragraph{Item b) El código muestra la sobrescritura del objeto en (a)
usando la misma secuencia con el orden
invertido.}\label{item-b-el-codigo-muestra-la-sobrescritura-del-objeto-en-a-usando-la-misma-secuencia-con-el-orden-invertido.}

\begin{Shaded}
\begin{Highlighting}[]
\NormalTok{v <-}\StringTok{ }\KeywordTok{seq}\NormalTok{(}\DataTypeTok{from=}\OperatorTok{-}\DecValTok{11}\NormalTok{,}\DataTypeTok{to=}\DecValTok{5}\NormalTok{,}\DataTypeTok{by=}\FloatTok{0.3}\NormalTok{)}
\NormalTok{v}
\end{Highlighting}
\end{Shaded}

\begin{verbatim}
##  [1] -11.0 -10.7 -10.4 -10.1  -9.8  -9.5  -9.2  -8.9  -8.6  -8.3  -8.0
## [12]  -7.7  -7.4  -7.1  -6.8  -6.5  -6.2  -5.9  -5.6  -5.3  -5.0  -4.7
## [23]  -4.4  -4.1  -3.8  -3.5  -3.2  -2.9  -2.6  -2.3  -2.0  -1.7  -1.4
## [34]  -1.1  -0.8  -0.5  -0.2   0.1   0.4   0.7   1.0   1.3   1.6   1.9
## [45]   2.2   2.5   2.8   3.1   3.4   3.7   4.0   4.3   4.6   4.9
\end{verbatim}

\paragraph{Item c) El código muestra la repitición del
vector}\label{item-c-el-codigo-muestra-la-repiticion-del-vector}

\begin{Shaded}
\begin{Highlighting}[]
\KeywordTok{c}\NormalTok{(}\OperatorTok{-}\DecValTok{1}\NormalTok{,}\DecValTok{3}\NormalTok{,}\OperatorTok{-}\DecValTok{5}\NormalTok{,}\DecValTok{7}\NormalTok{,}\OperatorTok{-}\DecValTok{9}\NormalTok{)}
\end{Highlighting}
\end{Shaded}

\begin{verbatim}
## [1] -1  3 -5  7 -9
\end{verbatim}

\paragraph{dos veces, con cada elemento repetido 10 veces y almacena el
resultado. Visualiza el resultado ordenado de mayor a
menor.}\label{dos-veces-con-cada-elemento-repetido-10-veces-y-almacena-el-resultado.-visualiza-el-resultado-ordenado-de-mayor-a-menor.}

\begin{Shaded}
\begin{Highlighting}[]
\NormalTok{s <-}\StringTok{ }\KeywordTok{rep}\NormalTok{ (}\KeywordTok{c}\NormalTok{ (}\OperatorTok{-}\DecValTok{1}\NormalTok{,}\DecValTok{3}\NormalTok{,}\OperatorTok{-}\DecValTok{5}\NormalTok{,}\DecValTok{7}\NormalTok{,}\OperatorTok{-}\DecValTok{9}\NormalTok{) , }\DecValTok{2}\OperatorTok{*}\DecValTok{10}\NormalTok{ )}
\KeywordTok{sort}\NormalTok{(s,}\DataTypeTok{decreasing =} \OtherTok{TRUE}\NormalTok{)}
\end{Highlighting}
\end{Shaded}

\begin{verbatim}
##   [1]  7  7  7  7  7  7  7  7  7  7  7  7  7  7  7  7  7  7  7  7  3  3  3
##  [24]  3  3  3  3  3  3  3  3  3  3  3  3  3  3  3  3  3 -1 -1 -1 -1 -1 -1
##  [47] -1 -1 -1 -1 -1 -1 -1 -1 -1 -1 -1 -1 -1 -1 -5 -5 -5 -5 -5 -5 -5 -5 -5
##  [70] -5 -5 -5 -5 -5 -5 -5 -5 -5 -5 -5 -9 -9 -9 -9 -9 -9 -9 -9 -9 -9 -9 -9
##  [93] -9 -9 -9 -9 -9 -9 -9 -9
\end{verbatim}

\subsubsection{Item d) El código muestra un vector con las siguientes
características:}\label{item-d-el-codigo-muestra-un-vector-con-las-siguientes-caracteristicas}

\paragraph{Secuencia con enteros del 6 al 12
(inclusive)}\label{secuencia-con-enteros-del-6-al-12-inclusive}

\paragraph{Repetición triple del valor
5.3}\label{repeticion-triple-del-valor-5.3}

\paragraph{El número -3}\label{el-numero--3}

\paragraph{Una secuencia de nueve valores que comienzan en 102 y termina
en el número que es la longitud total del vector creado en
(c)}\label{una-secuencia-de-nueve-valores-que-comienzan-en-102-y-termina-en-el-numero-que-es-la-longitud-total-del-vector-creado-en-c}

\begin{Shaded}
\begin{Highlighting}[]
\NormalTok{s1 <-}\StringTok{ }\KeywordTok{c}\NormalTok{(}\KeywordTok{seq}\NormalTok{(}\DataTypeTok{from=}\DecValTok{6}\NormalTok{,}\DataTypeTok{to=}\DecValTok{12}\NormalTok{),}\KeywordTok{rep}\NormalTok{(}\FloatTok{5.3}\NormalTok{,}\DecValTok{3}\NormalTok{),}\OperatorTok{-}\DecValTok{3}\NormalTok{,}\KeywordTok{seq}\NormalTok{(}\DataTypeTok{from=}\DecValTok{102}\NormalTok{,}\DataTypeTok{to=}\KeywordTok{length}\NormalTok{(s),}\DataTypeTok{length.out =} \DecValTok{9}\NormalTok{))}
\NormalTok{s1}
\end{Highlighting}
\end{Shaded}

\begin{verbatim}
##  [1]   6.00   7.00   8.00   9.00  10.00  11.00  12.00   5.30   5.30   5.30
## [11]  -3.00 102.00 101.75 101.50 101.25 101.00 100.75 100.50 100.25 100.00
\end{verbatim}

\paragraph{Confirma que la longitud del vector creado en (d) es
20}\label{confirma-que-la-longitud-del-vector-creado-en-d-es-20}

\begin{Shaded}
\begin{Highlighting}[]
\KeywordTok{length}\NormalTok{(s1)}
\end{Highlighting}
\end{Shaded}

\begin{verbatim}
## [1] 20
\end{verbatim}


\end{document}
